\documentclass[11pt,a4paper,titlepage]{article}

% Font related packages
\usepackage{fontspec}
\setmainfont{Lato}
\setsansfont{Lato}
\setmonofont{Fira Mono}
\newfontfamily\greekfont[Script=Greek]{Lato}

% Greek monospace font
\newfontfamily{\greekfonttt}{Fira Mono}
\usepackage{polyglossia}
\setdefaultlanguage{greek}

% Math packages
\usepackage{amsmath}
\usepackage{amsfonts}
\usepackage{amssymb}
\usepackage{makeidx}
\usepackage{eurosym}

% Graphics
\usepackage{graphicx}
\usepackage{float}
\usepackage{tikz}
\usetikzlibrary{arrows,positioning,shapes}

\usepackage{appendix} % Add appendix to document
\usepackage[margin=3cm]{geometry} % Change page margins

% Table packages
\usepackage{multirow}
\usepackage{color, colortbl}
\usepackage{array}
\usepackage{longtable}
\usepackage{lscape}

% PDF options
\usepackage[unicode,hidelinks,pdftitle={Ανάπτυξη συστήματος με υλοποίηση μεθοδολογίας MINORA (UTA) σε περιβάλλον Python με web interface},pdfauthor={Dimitrios Maroulidis, Myron Pachakis},pdfcreator={},pdfproducer={}]{hyperref}

% Colors to use in tables
\definecolor{LightGray}{gray}{0.95}
\definecolor{Gray}{gray}{0.9}
\definecolor{LightRed}{HTML}{ff6b6b}
%\definecolor{LightBlue}{HTML}{aacfff}
%\definecolor{LightBlue}{HTML}{5eaeff}
\definecolor{LightBlue}{HTML}{a7d2ff}

% Table column types for math environments
\newcolumntype{e}{>{$}c<{$}}
\renewcommand{\arraystretch}{1.2}
\usepackage{ragged2e}

% Math font
\usepackage{unicode-math}
\setmathfont{Asana Math}
% Use section based numbering for equations.
% amsmath is needed for this.
\numberwithin{equation}{section}

% Use comma as decimal point
%\usepackage{icomma}

% Hyperref
\usepackage{hyperref}

% Comments
\usepackage{comment}

% Bibliography with BibLaTeX
\usepackage[citestyle=alphabetic,bibstyle=alphabetic]{biblatex}
\addbibresource{bibliography.bib}

\author{Δημήτριος Μαρουλίδης (\textit{2018010111}) \and
    Μύρων Παχάκης (\textit{2018010028})}
\title{Ανάπτυξη συστήματος με υλοποίηση μεθοδολογίας MINORA (UTA) σε 
περιβάλλον Python με web interface}
\date{\today}

\begin{document}
\maketitle	
\tableofcontents
%	\listoffigures
%	\listoftables

\clearpage
% BEGIN HERE

\begin{comment}
Πλάνο Σύνταξης Αναφοράς
Εισαγωγή
1.Αντικειμενικοί Στόχοι Εργασίας-Προγράμματος
2.Περιγραφή μεθόδου θεωρητικά (Λογική της μεθόδου, ποιός την έφτιαξε)
3.Αναλύτικη Περιγραφή Μεθόδου (Χρησιμοποιούμε αυτό, κάνουμε αυτό, παρουσίαση αποτελέσματος, δεν σταματάμε όμως εκεί έχουμε ανατροφοδότηση από τον χρήστη (web interface), κάνει αυτή τη διαδικασία ο χρήστης)
4.Περιγραφή Υλοποίησης
5.Αναλυτικό Παράδειγμα εφαρμογής της μεθόδου
6.Παράρτημα -> Οδηγίες Χρήσης του προγράμματος-μεθόδου
7.Συμπεράσματα
8.Βιβλιογραφία
\end{comment}

\begin{abstract}
Παρουσιάζεται ένα αλληλεπιδραστικό πρόγραμμα στη γλώσσα προγραμματισμού Python για την επίλυση πολυκριτηρίων συστημάτων υποστήριξης αποφάσεων χρησιμοποιώντας το σύστημα MINORA. Για την υλοποίηση του συστήματος MINORA χρησιμοποιείται η μέθοδος πολυκριτήριας ανάλυσης UTASTAR.

Η εργασία χωρίζεται σε δύο μέρη, το θεωρητικό αποτελείται από τις ενότητες (1-3) και το προγραμματιστικό από τις ενότητες (4-7). Στην ενότητα \ref{sec:goals} παρουσιάζονται οι αντικειμενικοί στόχοι που αναμένονται από το πρόγραμμα. Στην ενότητα \ref{sec:theory} περιγράφεται αναλυτικά η θεωρία πίσω από την μέθοδο UTASTAR και το σύστημα MINORA, που περιλαμβάνει τη λογική λειτουργίας και τον τρόπο χρήσης τους, καθώς και τους πρωτοπόρους υλοποίησης και παρουσίασης των μεθόδων. Στην ενότητα \ref{sec:theory_in_depth} γίνεται αναλυτική παρουσίαση βήμα-βήμα της εφαρμογής των μεθόδων UTA-UTASTAR-MINORA.

Στην ενότητα \ref{sec:implementation} αναλύεται διεξοδικά το πρόγραμμα σε γλώσσα Python καθώς και η αλληλεπίδραση με τον χρήστη μέσω του web interface, ενώ στην ενότητα \ref{sec:example} δίνεται ένα παράδειγμα πολυκριτηρίων αποφάσεων και εφαρμόζεται η μέθοδος MINORA. Στην ενότητα \ref{sec:instructions} εντοπίζονται εκτενείς οδηγίες χρήσης του προγράμματος και στην ενότητα \ref{sec:conclusion} συνοψίζονται τα συμπεράσματα της εργασίας και της χρήσης των μεθόδων-προγράμματος. Στην ενότητα \ref{sec:bibliography} δίνεται η σχετική βιβλιογραφία και οι πηγές που χρησιμοποιήθηκαν.
\end{abstract}

\section{Αντικειμενικοί Στόχοι}
\label{sec:goals}   
\begin{itemize}
	\item Συγγραφή ενός προγράμματος σε γλώσσα προγραμματισμού Python με σκοπό την υποστήριξη για τη λήψη αποφάσεων από τον χρήστη.
	\item Υλοποίηση συστήματος MINORA της οποίας το αποτέλεσμα είναι ένα μοντέλο το οποίο εξομοιώνει τη διαδικασία απόφασης όσο πιο πιστά γίνεται.
	\item Παρουσίαση σε απλή και ευανάγνωστη μορφή του αποτελέσματος.
	\item Δυνατότητα άμεσης αλληλεπίδρασης του χρήστη με το πρόγραμμα μέσω web interface.
	\item Επίδειξη αναλυτικών οδηγιών χρήσης του προγράμματος.
\end{itemize}


\section{Θεωρητική Περιγραφή Μεθόδου}
\label{sec:theory}
Η μέθοδος πολυκριτήριας ανάλυσης UTASTAR (UTilités Additives *) είναι υπεύθυνη για την εκτίμηση μίας ή παραπάνω συναρτήσεων αξίας χρησιμοποιώντας διαδικασίες γραμμικού προγραμματισμού. Στόχος των συναρτήσεων αξίας που θα εξαχθούν είναι η όσο το δυνατόν καλύτερη κατάταξη κάποιων επιλογών $a$ σε σύγκριση με την αρχική κατάταξη που έχει δοθεί από τον χρήστη. Ο χρήστης είναι υπεύθυνος για την εισαγωγή ενός συνόλου επιλογών $A_{R}$ διατεταγμένου κατά σειρά προτίμησης καθώς επίσης και ενός συνόλου κριτηρίων επιλογής $g_{n}$. 

Το σύστημα MINORA (Multicriteria Interactive Ordinal Regression Analysis), χρησιμοποιεί την μέθοδο UTASTAR για την εξαγωγή της λύσης του πολυκριτηρίου προβλήματος, με την διαφορά ότι το σύστημα δίνει την δυνατότητα στον χρήστη να ανατροφοδοτήσει τυχόν δεδομένα, καταλήγοντας στο επιθυμητό σύστημα προτιμήσεων.

Πρωτοπόροι επιμελητές της μεθόδου UTASTAR είναι οι καθηγητές Siskos, Y. και Yannacopoulos, D., το 1985 \cite{utastar}, ενώ το σύστημα MINORA επιμελήθηκε από τον καθηγητή Siskos, Y. και συνεργάτες, το 1993 \cite{minora} και το 1994 \cite{SISKOS1994151}.

\newpage

\section{Αναλυτική Περιγραφή Μεθόδου}
\label{sec:theory_in_depth}
\subsection{UTASTAR}
Όπως είδαμε παραπάνω ο χρήστης παρέχει ένα σύνολο $A_{R}$ εναλλακτικών επιλογών $a$ ταξινομημένων κατά σειρά προτίμησης από το καλύτερο στο χειρότερο καθώς επίσης και ένα σύνολο κριτηρίων $g_{n}$. Ανάλογα με τον αριθμό των επιλογών, θα χωριστούν ισομήκη διαστήματα για κάθε κριτήριο χωριστά: 

\begin{equation}\label{initial_inter}
[g_{i*},g^{*}_{i}] = [g^{1}_{i}, ..., g^{a_{i}}_{i}]
\end{equation}
όπου $g^{a_{i}}_{i}$ οι διάφορες τιμές του συγκεκριμένου κριτηρίου, $g_{i*}$ η χειρότερη τιμή του και $g^{*}_{i}$ η καλύτερη τιμή του.
Σε αυτό το σημείο είναι απαραίτητο να γίνει ορισμός της έννοιας των χρησιμοτήτων των διαφόρων επιλογών $u(g(a))$, ο οποίος αποτελεί το μοντέλο απόφασης στην μέθοδο UTASTAR. Εκφράζεται ως μία προσθετική συνάρτηση των αυξουσών μερικών χρησιμοτήτων των κριτηρίων $u_{i}(g^{j}_{i})$:
\begin{equation}\label{total_uti}
	u(g(a)) = \sum_{i=1}^{n} u_{i}(g_{i}(a)) = u_{1}(g_{1}(a)) + u_{2}(g_{2}(a)) + ... + u_{n}(g_{n}(a))
\end{equation}
όπου $i = 1,2,...,n$ ο αριθμός του κριτηρίου και $a\in A_{R}$ η συγκεκριμένη επιλογή.

Επειδή ο χρήστης έχει κατατάξει τις επιλογές του κατά σειρά προτίμησης, είναι προφανές ότι η επιλογή $a_{m}$ θα είναι πάντα προτιμότερη από ή ίση (αδιάφορη) με την επιλογή $a_{m+1}$.

Είναι, λοιπόν, απαραίτητο να διατηρηθεί η μονοτονία των κριτηρίων $g_{i}$. Έτσι με βάση την προτιμησιακή ανεξαρτησία των κριτηρίων η μονοτονία παραμένει ίδια τόσο στις μερικές συναρτήσεις χρησιμοτήτων $u_{i}(g^{j}_{i})$ όσο και στην ολική συνάρτηση  $u(g(a))$.

\centerline{Σε περίπτωση προτίμησης ισχύει: $u_{i}(g^{j+1}_{i}) > u_{i}(g^{j}_{i})$}

\centerline{Σε περίπτωση αδιαφορίας ισχύει: $u_{i}(g^{j+1}_{i}) = u_{i}(g^{j}_{i})$}

Στη συνέχεια ορίζονται οι μεταβλητές $w_{ij}$ σύμφωνα με τις παραπάνω σχέσεις μονοτονίας των κριτηρίων ως:

\begin{equation}\label{def_w}
	w_{ij} = u_{i}(g^{j+1}_{i}) - u_{i}(g^{j}_{i}) \geq 0
\end{equation}
όπου $i = 1,2,...,n$ και $j = 1,2,...,a_{i}-1$. Με αυτόν τον τρόπο είναι δυνατόν να αναπαρασταθεί η εξίσωση \ref{total_uti} των ολικών χρησιμοτήτων $u(g(a))$ με χρήση των μεταβλητών $w_{ij}$ αντί των μερικών χρησιμοτήτων $u_{i}(g^{j}_{i})$. 

Για να πραγματοποιηθεί αυτή η μετατροπή είναι απαραίτητο να χρησιμοποιηθούν οι παρακάτω σχέσεις:

\begin{equation}\label{zero_crit}
	u_{i}(g^{1}_{i}) = 0
\end{equation}

\begin{equation}\label{sum_w}
	u_{i}(g^{j}_{i}) = \sum_{i=1}^{j-1} w_{ij}
\end{equation}
όπου $i = 1,2,...,n$ και $j = 2,3,...,a_{i}-1$.

\newpage

Η εξίσωση \ref{zero_crit} είναι υπεύθυνη για τον μηδενισμό της χειρότερης τιμής του κάθε κριτηρίου ενώ στην εξίσωση \ref{sum_w} παρουσιάζεται η τιμή της χρησιμότητας του κάθε κριτηρίου ως το άθροισμα των μεταβλητών $w_{ij}$.  

Εάν οποιαδήποτε μερική χρησιμότητα $u_{i}(g_{i}(a))$ δεν υπάρχει στις διακριτές τιμές του αντίστοιχου κριτηρίου τότε εφαρμόζεται γραμμική παρεμβολή στο συγκεκριμένο κριτήριο $g_{i}^{j}$. 

Θεωρούνται πεπερασμένα τα άκρα $g_{i*}$ και $g^{*}_{i}$ του κριτηρίου και γίνεται χωρισμός του διαστήματος από την σχέση \ref{initial_inter} σε ισομήκη υποδιαστήματα $(a_{i}-1)$, όπου το συγκεκριμένο σημείο που θα γίνει ο χωρισμός υπολογίζεται από τον τύπο:

\begin{equation}
	g_{i}^{j} = g_{i*} + \frac{j-1}{a_{i}-1}(g^{*}_{i}-g_{i*})
\end{equation}
για $j = 1,2,...,a_{i}$ και $g_{i*}$ η χειρότερη τιμή του συγκεκριμένου κριτηρίου και $g^{*}_{i}$ η καλύτερη τιμή του συγκεκριμένου κριτηρίου.

Η γραμμική παρεμβολή εφαρμόζεται σύμφωνα με τον τύπο \ref{linear_inter} και δίνει ως αποτέλεσμα την τιμή της μερικής χρησιμότητας για την επιλογή $a$.

\begin{equation}\label{linear_inter}
	u_{i}(g_{i}(a)) = u_{i}(g^{j}_{i}) + 
	\frac{g_{i}(a)-g^{j}_{i}}{g^{j+1}-g^{j}_{i}}(u_{i}(g_{i}^{j+1})-u_{i}(g_{i}^{j}))
\end{equation}


Στη μέθοδο UTASTAR είναι απαραίτητο να χρησιμοποιηθεί μία διπλή συνάρτηση σφάλματος που θα εκφράζει το ποσό της χρησιμότητας που θα πρέπει να προστεθεί ή αφαιρεθεί, αντίστοιχα, από την ολική χρησιμότητα $u(g(a))$ ώστε να είναι δυνατόν η επιλογή $a$ που έχει ως ολική χρησιμότητα την $u(g(a))$ να ανακατακτήσει την θέση της στη διάταξη των επιλογών.

Τα δύο σφάλματα αυτά ονομάζονται σφάλμα υποεκτίμησης $σ^{+}$ το οποίο αφαιρεί ένα ποσό αξίας και σφάλμα υπερεκτίμησης $σ^{-}$ το οποίο προσθέτει ένα ποσό αξίας.

Με αυτόν τον τρόπο τα σημεία των επιλογών $a$ σταθεροποιούνται με μεγαλύτερη ακρίβεια πάνω στην καμπύλη μονότονης παλινδρόμησης όπως παρουσιάζεται στο σχήμα \ref{fig:graph_monot}.

\begin{figure}[H]
	\centering
	\includegraphics[width=0.7\linewidth]{media/graph_mono.jpg}
	\caption{Καμπύλη Μονότονης Παλινδρόμησης}
	\label{fig:graph_monot}
\end{figure}

Με τις παραπάνω προσθήκες των σφαλμάτων η εξίσωση των ολικών χρησιμοτήτων \ref{total_uti} μετατρέπεται σε:

\begin{equation}\label{total_uti_fin}
	u(g(a)) = \sum_{i=1}^{n} u_{i}(g_{i}(a)) - σ^{+}(a) + σ^{-}(a)
\end{equation}
όπου $i = 1,2,...,n$ ο αριθμός του κριτηρίου και $a\in A_{R}$ η συγκεκριμένη επιλογή.

Έτσι για δύο διαδοχικές επιλογές $a_{m}$ και $a_{m+1}$ ορίζεται:

\begin{equation}\label{def_deltas}
	Δ(a_{m}, a_{m+1}) = u(g(a_{m})) - σ^{+}(a_{m}) + σ^{-}(a_{m}) - u(g(a_{m+1})) + σ^{+}(a_{m+1}) - σ^{-}(a_{m+1})
\end{equation}

Όπου οι χρησιμότητες $u(g(a_{m}))$ μπορούν να αντικατασταθούν από τις μεταβλητές $w_{ij}$ με βάση τις σχέσεις \ref{def_w} και \ref{sum_w}.
 
Έχει ήδη αναφερθεί πως η μέθοδος UTASTAR χρησιμοποιεί μεθόδους γραμμικού προγραμματισμού για τον υπολογισμό των μερικών συναρτήσεων χρησιμότητας-αξίας. Επιλύεται το παρακάτω γραμμικό πρόβλημα (ΓΠ) με αντικειμενική συνάρτηση ελαχιστοποίησης των συνολικών σφαλμάτων $σ^{+}$ και $σ^{-}$.

\begin{equation}\label{linear_program}
	\begin{aligned}
	&\text{min}~F = \sum_{a\in A_{R}} {σ^{+}(a) + σ^{-}(a)} \\
	&\text{υπό περιορισμούς}\\
	&Δ(a_{m},a_{m+1}) \geq δ ~\text{εάν}~ a_{m}\succ a_{m+1}\\
	&Δ(a_{m},a_{m+1}) = 0 ~\text{εάν}~ a_{m}\sim a_{m+1}\\
	&\sum_{n}^{i=1}\sum_{j=1}^{a_{i}-1} w_{ij} = 1\\
	& w_{ij} \geq 0, σ^{+}(a) \geq 0, σ^{-}(a) \geq 0 \forall a\in A_{R}, \forall i,j
	\end{aligned}
\end{equation}
 
\newpage

Όπου $δ$ ορίζεται ως η τιμή του κατωφλιού προτίμησης και έχει μία μικρή θετική τιμή. Εξασφαλίζει ότι η διάταξη των εναλλακτικών ταιριάζει με το μοντέλο προτιμήσεων του αποφασίζοντα.

Αφού εξαχθεί η λύση του γραμμικού προγράμματος τότε γίνεται ανάλυση ευστάθειας, δηλαδή ελέγχουμε εάν υπάρχουν περισσότερες από μία βέλτιστες λύσεις ή εάν υπάρχουν λύσεις που είναι πολύ κοντά στη βέλτιστη που έχουμε υπολογίσει.

Σε περίπτωση που η βέλτιστη λύση της αντικειμενικής συνάρτησης έχει τιμή μηδέν τότε υπάρχουν παραπάνω από μία λύσεις που ικανοποιούν την διάταξη επιλογών $A_{R}$ που έχει εισάγει ο χρήστης. Τότε υπολογίζουμε τη μέση τιμή των συναρτήσεων χρησιμότητας εκείνων των κοντινότερων βέλτιστων λύσεων που μεγιστοποιούν τις αντικειμενικές συναρτήσεις:

\begin{equation}\label{eq:sens-analysis-Obj-Fun}
	u_{i}(g_{i}^{*}) = \sum_{j=1}^{a_{i}-1} w_{ij}  ~~\forall i=1,2,..,n
\end{equation} 

Η αντικειμενική συνάρτηση του αρχικού γραμμικού προγράμματος \eqref{linear_program} θέτεται ως περιορισμός:

\begin{equation}\label{eq:sens-analysis-constr}
	\sum_{a\in A_{R}} {σ^{+}(a) + σ^{-}(a)} \leq z^{*} + ε
\end{equation}

Όπου $z_{*}$ είναι η βέλτιστη τιμή του αρχικού γραμμικού προγράμματος \ref{linear_program}, δηλαδή το ελάχιστο σφάλμα, και $ε$ είναι είτε μηδέν είτε ένας μικρός θετικός αριθμός. 

Όταν υπάρχει αστάθεια οι βέλτιστες λύσεις των γραμμικών προγραμμάτων παρουσιάζουν μεγάλη απόκλιση μεταξύ τους, γεγονός το οποίο οδηγεί σε μεγαλύτερη διακύμανση των μεταβλητών $w_{ij}$. Με αυτόν τον τρόπο μπορούμε να καταλάβουμε πόσο σημαντικό είναι το κάθε κριτήριο $g_{i}$ στη διάταξη επιλογών που έχει ορίσει ο χρήστης.

Σε αυτό το σημείο έχει εφαρμοστεί η μέθοδος UTASTAR και έχουν εξαχθεί τα τελικά αποτελέσματα καθώς και η τελική κατάταξη των εναλλακτικών που προτείνει το σύστημα σύμφωνα με τη συνάρτηση χρησιμότητας.
Επιπλέον σημαντικό είναι να οριστεί μία μεταβλητή η οποία θα προσδιορίζει την διαφορά της προδιάταξης του χρήστη από την τελική διάταξη που προκύπτει από τη συνάρτηση χρησιμότητας. Αυτή η μεταβλητή ονομάζεται $τ$ του Kendall και κυμαίνεται στο διάστημα $[-1,1]$ όπου η τιμή $-1$ σημαίνει πως δεν υπάρχει καμία ομοιότητα ανάμεσα στις δύο κατατάξεις και η τιμή $1$ σημαίνει πως οι κατατάξεις ταυτίζονται.

\subsection{MINORA}
\label{ssec:minora-theory}
Όπως έχει ήδη αναφερθεί το σύστημα MINORA χρησιμοποιεί αυτά τα δεδομένα-αποτελέσματα και επιτρέπει στον χρήστη να προβεί σε διάφορες ενέργειες-αναδράσεις έτσι ώστε να υπάρχει απόλυτη συμφωνία ανάμεσα στη αρχική προδιάταξη και στη διάταξη του μοντέλου.

Παρακάτω συνοψίζονται οι διαφορετικές αναδράσεις με τις οποίες ο χρήστης μπορεί να επιτύχει απόλυτη συμφωνία. Η προτίμηση κάποιας εκ των αναδράσεων είναι αποκλειστικά επιλογή του χρήστη-αποφασίζοντα.

\begin{description}
	\item[Ανάδραση τύπου 1]
	Ο χρήστης αρνείται να μεταβάλει την σειρά προτίμησης των εναλλακτικών ώστε να ταυτίζεται με το τελικό μοντέλο και συνεπώς διαφοροποιεί τη μοντελοποίηση των κριτηρίων:
	\begin{itemize}
		\item Είτε ενώνοντας δύο παρόμοια κριτήρια σε ένα κοινό.
		\item Είτε χωρίζοντας ένα σύνθετο κριτήριο σε δύο.
		\item Είτε αλλάζοντας το διάστημα που κυμαίνεται ένα ή περισσότερα κριτήρια.
		\item Είτε προσθέτοντας καινούργιο κριτήριο ή κριτήρια.
		\item Είτε διαγράφοντας ένα ή και περισσότερα κριτήρια.
	\end{itemize}
	Σε κάθε μία από τις παραπάνω περιπτώσεις ο αποφασίζων επαναχρησιμοποιεί τη μέθοδο UTASTAR για την επίλυση του καινούργιου προβλήματος με τις διαφορετικές μεταβλητές.
	\item[Ανάδραση τύπου 2]
	Σε αυτή την περίπτωση ο χρήστης θεωρεί σωστό το τελικό μοντέλο και αλλάζει την κρίση του έτσι ώστε να ταυτίζεται με τα αποτελέσματα του μοντέλου.	
	\item[Ανάδραση τύπου 3]
	Παρομοίως με την ανάδραση τύπου 1 ο χρήστης διαφωνεί με το τελικό μοντέλο κατάταξης αντί όμως να αλλάξει κριτήρια, μεταβάλει τις μερικές χρησιμότητες των κριτηρίων στα σημεία όπου είναι δυνατόν να πραγματοποιηθεί αυτό. Όπως είναι λογικό θα προκύψει διαφορετική συνάρτηση χρησιμότητας και ο χρήστης εφαρμόζει ξανά την μέθοδο UTASTAR για να ελέγξει τυχών νέα ασυμφωνία.   
	\item[Ανάδραση τύπου 4]
	Όταν δεν υπάρχει καμία απολύτως συμφωνία ανάμεσα στο τελικό μοντέλο και στη προδιάταξη τότε είναι απαραίτητο να απορριφθεί το τρέχον μοντέλο ανάλυσης και να δημιουργηθεί ένα καινούργιο από την αρχή.
	\item[Ανάδραση τύπου 5]
	Η πλήρης συμφωνία ανάμεσα στο μοντέλο και την προδιάταξη του χρήστη οδηγεί σε εφαρμογή της βέλτιστης συνάρτησης χρησιμότητας για αξιολόγηση επιπλέον επιλογών-εναλλακτικών. Εάν μετά από προσθήκη νέων εναλλακτικών και εφαρμογή της UTASTAR υπάρχει ασυμφωνία τότε θα πρέπει να τροποποιηθεί το αρχικό πρόβλημα και να εφαρμοστεί ξανά η UTASTAR.
\end{description}

\section{Περιγραφή Υλοποίησης}
\label{sec:implementation}
Η εφαρμογή αποτελείται από δύο μέρη: τον πυρήνα που υλοποιεί τη μέθοδο UTASTAR και τη διεπαφή ιστού (web interface) που υλοποιεί τη μέθοδο MINORA και είναι το κομμάτι της εφαρμογής με το οποίο αλληλεπιδρά ο χρήστης. Παρακάτω περιγράφουμε τα μέρη αυτά, τον τρόπο λειτουργίας και αλληλεπίδρασής τους.

\subsection{Πυρήνας}
\label{ssec:impl-kernel}
Ο κώδικας του πυρήνα, που υλοποιεί τη μέθοδο UTASTAR, βρίσκεται στον φάκελο \textbf{algorithms}. Το μοναδικό αρχείο αυτού του φακέλου περιέχει την συνάρτηση \hyperref[sssec:utastar()]{\texttt{utastar()}}, που υλοποιεί τον ομώνυμο αλγόριθμο, μαζί με άλλες βοηθητικές κλάσεις. Συγκεκριμένα το αρχείο \textbf{utastar.py} περιέχει τις εξής κλάσεις, με σειρά εμφάνισης:

\begin{itemize}
    \item \texttt{LinearProgramError(Exception)}
    \item \texttt{Subinterval}
    \item \texttt{Interval(Subinterval)}
    \item \texttt{Criterion}
    \item \texttt{Criteria}
    \item \texttt{UtastarResult}
\end{itemize}

\subsubsection{Κλάση \texttt{LinearProgramError(Exception)}}
\label{sssec:LinearProgramError}
Αυτή η κλάση κληρονομεί τη βασική κλάση (base class) \texttt{Exception} και χρησιμοποιείται σε περίπτωση σφάλματος κατά την επίλυση γραμμικών προγραμμάτων, όπως αυτά που χρησιμοποιεί η UTASTAR κατά την μοντελοποίηση του προβλήματος.

\subsubsection{Κλάση \texttt{Subinterval}}
\label{sssec:Subinterval}
Η κλάση χρησιμοποιείται για να αναπαραστήσει ένα υποδιάστημα του διαστήματος 
τιμών ενός κριτηρίου. Για τη δημιουργία του χρειάζεται δύο δεκαδικούς αριθμούς, 
τα δύο άκρα του υποδιαστήματος.

\subsubsection{Κλάση \texttt{Ιnterval(Subinterval)}}
\label{sssec:Ιnterval}
Η κλάση \texttt{Ιnterval(Subinterval)} αναπαριστά το διάστημα τιμών ενός κριτηρίου. Αυτό αποτελείται από μια λίστα υποδιαστημάτων (αντικείμενα \texttt{Subinterval}). Για τη δημιουργία του χρειάζεται τα δύο άκρα του διαστήματος τιμών του κριτηρίου και τον αριθμό υποδιαστημάτων στα οποία αυτό χωρίζεται.

\subsubsection{Κλάση \texttt{Criterion}}
\label{sssec:Criterion}
Η κλάση \texttt{Criterion} χρησιμοποιείται στη μοντελοποίηση των κριτηρίων. Αναπαριστά ένα κριτήριο και έχει δύο ιδιότητες: το όνομα του κριτηρίου και το διάστημα τιμών του (το οποίο είναι αντικείμενο της κλάσης \texttt{Interval}).

Η κλάση έχει και μία σημαντική μέθοδο, τη 
\hypertarget{method:getvalue}{\texttt{get\_value()}}, η οποία είναι υπεύθυνη 
για τον υπολογισμό των μερικών χρησιμοτήτων των εναλλακτικών για κριτήριο που 
μοντελοποιείται από την κλάση. Η \texttt{get\_value()} λαμβάνει ως είσοδο την 
τιμή μιας εναλλακτικής στο κριτήριο και επιστρέφει μια λίστα με τους 
συντελεστές πολλαπλασιασμού των διαφορών μερικών βαρών των υποδιαστημάτων 
(τιμές $ w_{ij} $). Tο εσωτερικό γινόμενο της λίστας αυτής με τα βάρη ($ w_{ij} 
$), είναι η μερική χρησιμότητα της εναλλακτικής στο συγκεκριμένο κριτήριο.

\subsubsection{Κλάση \texttt{Criteria}}
\label{sssec:Criteria}
Η κλάση αναπαριστά την συλλογή κριτηρίων προβλήματος. Έχει μια μέθοδο, την \texttt{weight\_array()}, που χρησιμοποιείται για την κατασκευή των διανυσμάτων συντελεστών των αντικειμενικών συναρτήσεων των γραμμικών προγραμμάτων εύρεσης των καλύτερων μερικών βαρών των κριτηρίων, στην περίπτωση πολλαπλότητας λύσεων στο αρχικό γραμμικό πρόγραμμα \eqref{linear_program}.

\subsubsection{Κλάση \texttt{UtastarResult}}
\label{sssec:UtastarResult}
Αποθηκεύει τα αποτελέσματα μιας εκτέλεσης της συνάρτησης \hyperref[sssec:utastar()]{\texttt{utastar()}}, και περιέχει μερικές βοηθητικές μεθόδους για τη χρήση των αποτελεσμάτων σε άλλους υπολογισμούς. 

Συγκεκριμένα όταν χρησιμοποιείται η κλάση δημιουργείται ένα αντικείμενο της 
κλάσης που παριστά την τελική λύση του προβλήματος. Στην περίπτωση εκτέλεσης 
της ανάλυσης ευαισθησίας το αντικείμενο επεκτείνεται και ενσωματώνει ένα 
αντικείμενο \texttt{UtastarResult} που παριστά τη λύση του γραμμικού 
προγράμματος \eqref{linear_program} (στην ιδιότητα \texttt{first\_sol}) και μία 
λίστα αντικειμένων \texttt{UtastarResult} που παριστούν τις λύσεις των 
γραμμικών προγραμμάτων των κριτηρίων στην ανάλυση ευαισθησίας (στην ιδιότητα 
\texttt{sa\_sol}), ένα για κάθε λύση.

Κάθε αντικείμενο περιέχει τις τιμές $ w_{ij} $ των κριτηρίων, τον πολυκριτήριο 
πίνακα, τις χρησιμότητες των εναλλακτικών, τα σφάλματα υπερ/υποεκτίμησης των 
αξιών των εναλλακτικών και τον αριθμό $ τ $ του Kendall.

Η κλάση έχει δύο μεθόδους: την \texttt{get\_utility()} και την 
\texttt{get\_crit\_weights}. H πρώτη, λαμβάνοντας ως είσοδο τη λίστα με τις 
τιμές μια νέας εναλλακτικής στα κριτήρια του προβλήματος, υπολογίζει την ολική 
χρησιμότητα της εναλλακτικής. Η δεύτερη επιστρέφει τις διαφορές των μερικών 
βαρών (τιμές $ w_{ij} $) του κριτηρίου για το οποίο κλήθηκε.

\subsubsection{Συνάρτηση \texttt{utastar()}}
\label{sssec:utastar()}
H συνάρτηση \texttt{utastar()} είναι υλοποιεί τον αλγόριθμο UTASTAR. H συνάρτηση λαμβάνει ως είσοδο πέντε μεταβλητές:
\begin{description}
    \item[\texttt{multicrit\_tbl (pandas.DataFrame)}]  
    Ο πολυκριτήριος πίνακας του προβλήματος. Περιέχει τα δεδομένα του προβλήματος σε συγκεκριμένες στήλες: η πρώτη στήλη περιέχει τα ονόματα των εναλλακτικών, η δεύτερη στήλη την προδιάταξή τους, και οι υπόλοιπες στήλες περιέχουν τις τιμές των εναλλακτικών στα κριτήρια και υπάρχει μία στήλη για κάθε κριτήριο.
    \item[\texttt{crit\_monot (dict)}]
    Λεξικό με τα ονόματα κριτηρίων (όπως είναι στις στήλες του πολυκριτηρίου πίνακα) και δυαδικές τιμές (\texttt{True} και \texttt{False}) που δείχνουν την μονοτονία των κριτηρίων. Η τιμή \texttt{True} αντιστοιχεί σε αύξον και η τιμή \texttt{False} σε φθίνον κριτήριο.
    \item[\texttt{a\_split (dict)}]
    Λεξικό με τα ονόματα κριτηρίων (όπως είναι στις στήλες του πολυκριτηρίου πίνακα) και τον αριθμό υποδιαστημάτων στα οποία θα χωριστούν τα διαστήματα τιμών των κριτηρίων.
    \item[\texttt{delta (float)}]
    Η τιμή του κατωφλιού προτίμησης.
    \item[\texttt{epsilon (float)}]
    Η τιμή $ \varepsilon $ της μεθόδου UTASTAR.
\end{description}

Αρχικά η συνάρτηση ταξινομεί τον πολυκριτήριο πίνακα κατά αύξουσα σειρά και βρίσκει τη μέγιστη και ελάχιστη τιμή κάθε κριτηρίου. Χρησιμοποιεί τις τιμές αυτές για να δημιουργήσει μια λίστα με αντικείμενα κλάσης
\hyperref[sssec:Criterion]{\texttt{Criterion}} στα οποία περνάει τα ακρότατα και τη μονοτονία κάθε κριτηρίου ώστε να υπολογίσουν το διάστημα τιμών τους. Μετά κατασκευάζει ένα αντικείμενο κλάσης \hyperref[sssec:Criteria]{\texttt{Criteria}} από τη λίστα των κριτηρίων. 
Η επεξεργασία συνεχίζεται με τη δημιουργία λίστας διανυσμάτων που περιέχουν 
τους συντελεστές πολλαπλασιασμού των διαφορών μερικών βαρών ($ w_{ij} $), που 
επιστρέφει η μέθοδος \hyperlink{method:getvalue}{\texttt{get\_value}}. Στις 
επόμενες 5 γραμμές κατασκευάζεται η λίστα των συντελεστών από τις διαφορές των 
εναλλακτικών σύμφωνα με την κατάταξή τους και τις σχέσεις προτίμησης / 
αδιαφορίας. Μετά δημιουργείται ο πίνακας με τις τιμές $ σ^{+} $ και $ σ^{-} $ 
της συνάρτησης διπλού σφάλματος για κάθε διαφορά.
 
Οι πίνακες διαφορών και σφαλμάτων συνενώνονται και αποτελούν το αριστερό μέρος των περιορισμών του γραμμικού προγράμματος \eqref{linear_program}. Στις επόμενες 45 γραμμές γίνεται διαχωρισμός των αριστερών και δεξιών μερών των περιορισμών, κατασκευάζεται η αντικειμενική συνάρτηση ελαχιστοποίησης των συνολικών σφαλμάτων $ σ^{+} $ και $ σ^{-} $ και επιλύεται το γραμμικό πρόγραμμα.

Με την επίλυση του ΓΠ ξεκινά η ανάλυση ευστάθειας της λύσης. Εάν η βέλτιστη 
τιμή της αντικειμενικής συνάρτησης είναι πολύ κοντά στο μηδέν, τότε υπάρχουν 
πολλαπλές βέλτιστες λύσεις. Σε αυτήν την περίπτωση η συνάρτηση βρίσκει τις 
πλησιέστερες βέλτιστες λύσεις που μεγιστοποιούν τις αντικειμενικές συναρτήσεις 
\eqref{eq:sens-analysis-Obj-Fun} υπό τους περιορισμούς 
\eqref{eq:sens-analysis-constr} και παίρνει το μέσο όρο των διαφορών μερικών 
βαρών των βέλτιστων λύσεων. 

Τελικά δημιουργεί και επιστρέφει ένα αντικείμενο κλάσης \hyperref[sssec:UtastarResult]{\texttt{UtastarResult}}
που περιέχει: το αντικείμενο \hyperref[sssec:Criteria]{\texttt{Criteria}}, τις 
τιμές των διαφορών μερικών βαρών ($ w_{ij} $) των κριτηρίων, την λίστα των 
συντελεστών τους, τα σφάλματα υπο/υπερεκτίμησης και τον πολυκριτήριο πίνακα.


\subsection{Διεπαφή ιστού (web interface)}
\label{ssec:web-interface}
Το κομμάτι του προγράμματος με το οποίο αλληλεπιδρά ο χρήστης, δηλαδή η διεπαφή ιστού, χρησιμοποιεί το πρότζεκτ \href{https://www.djangoproject.com/}{Django}. Όλη η διεπαφή έχει υλοποιηθεί ως μία εφαρμογή ονόματι minora η οποία βρίσκεται στον ομώνυμο φάκελο. Οι ρυθμίσεις του ιστοτόπου, δηλαδή του Django project σύμφωνα με την ορολογία του Django, βρίσκονται στον φάκελο \texttt{bluehunt}.

Η εφαρμογή minora (\textbf{minora/}) υλοποιείται από τα εξής αρχεία και φακέλους:
\begin{description}
\item[models.py] Εδώ ορίζονται τα \href{https://docs.djangoproject.com/en/3.2/topics/db/models/}{μοντέλα (Models)} που χρειάζεται το Django. Στην εφαρμογή υπάρχει ένα μοντέλο, το \texttt{Problem} που αναπαριστά ένα πρόβλημα Minora προς επίλυση. Έτσι αποθηκεύονται στη βάση δεδομένων πληροφορίες του προβλήματος που χρησιμοποιούνται από την εφαρμογή.
\item[views.py] Εδώ ορίζεται ο κώδικας που δημιουργεί τις σελίδες (\href{https://docs.djangoproject.com/en/3.2/topics/http/views/}{views}) που επισκέπτεται ο περιηγητής του χρήστη.
\item[templates/] Ο φάκελος περιέχει τα πρότυπα HTML (HTML templates) που χρησιμοποιεί ο κώδικας των views για να φτιάξει τις σελίδες για τον περιηγητή.
\item[forms.py] Κώδικας που χρησιμοποιείται για τη δημιουργία και χρήση των πεδίων φόρμας που υπάρχουν σε ορισμένες σελίδες της εφαρμογής.
\item[urls.py] Εδώ συνδέονται τα URL (οι διευθύνσεις) της εφαρμογής με τον κώδικα (views) που παράγει τις σελίδες που επισκέπτεται ο χρήστης.
\end{description}

\subsection{Models}
\label{ssec:models}
Όπως είδαμε παραπάνω τα models της εφαρμογής ορίζονται στο αρχείο \textbf{models.py}. Εκεί υπάρχει το μοναδικό 
μοντέλο της εφαρμογής, το \texttt{Problem} που αναπαριστά ένα πρόβλημα Minora προς επίλυση. Το μοντέλο έχει τέσσερα πεδία (\href{https://docs.djangoproject.com/en/3.2/ref/models/fields/}{fields}): ένα για το όνομα, ένα για το αρχείο Excel με τα δεδομένα του προβλήματος και δύο για τους αριθμούς $ \delta $ και $ \varepsilon $.

Το μοντέλο \texttt{Problem} έχει, επίσης, δύο μεθόδους που βοηθούν με τη φόρτωση του αρχείου Excel (μέθοδος \texttt{get\_dataframe()}) και φροντίζουν για τη σωστή εκτέλεση της \hyperref[sssec:utastar()]{\texttt{utastar()}} με τα δεδομένα του προβλήματος (μέθοδος \texttt{run\_utastar()}).

\subsection{Views}
\label{ssec:views}
Τα \href{https://docs.djangoproject.com/en/3.2/topics/http/views/}{views} είναι συναρτήσεις ή κλάσεις που δημιουργούν τη σελίδα που βλέπει ο χρήστης στον περιηγητή του. Αυτές ορίζονται στο αρχείο \textbf{views.py}, όπως είδαμε παραπάνω. Εκεί υπάρχουν εννέα \href{https://docs.djangoproject.com/en/3.2/topics/http/views/}{views} συνολικά που υλοποιούν όλες τις διαδικασίες που υπαγορεύει η μέθοδος Minora.

\begin{description}
\item[\hypertarget{IndexView}{\texttt{IndexView}}] View υπεύθυνο για τη δημιουργία της αρχικής σελίδας της εφαρμογής, η 
οποία εμφανίζει τη λίστα με τα εισαγμένα προβλήματα στην εφαρμογή. Από εδώ ο χρήστης μπορεί να
κατευθυνθεί στην \hyperlink{details}{σελίδα των πληροφοριών του προβλήματος}, στην \hyperlink{results}{σελίδα αποτελεσμάτων UTASTAR} ή να διαγράψει το πρόβλημα.
\item[\hypertarget{uploadproblem}{\texttt{upload\_problem}}] View υπεύθυνο για την εισαγωγή νέου προβλήματος στην εφαρμογή.
\item[\hypertarget{replaceproblemfile}{\texttt{replace\_problem\_file}}] View υπεύθυνο για την αντικατάσταση του αρχείου Excel του προβλήματος. Υποστηρίζει τις αναδράσεις τύπου 1 και 4 (υποενότητα \ref{ssec:minora-theory}) του χρήστη.
\item[\hypertarget{details}{\texttt{details}}] Εμφανίζει τον πολυκριτήριο πίνακα και τις ιδιότητες των κριτηρίων όπως ερμηνεύθηκαν από το αρχείο Excel. Επιτρέπει τον ορισμό των παραμέτρων $ \delta $ και $ \varepsilon $, με τους οποίους μπορεί ο χρήστης να πραγματοποιήσει και την ανάδραση τύπου 3 (υποενότητα \ref{ssec:minora-theory}) σε περίπτωση ασυμφωνίας αποτελεσμάτων - κρίσης χρήστη.
\item[\hypertarget{results}{\texttt{results}}] Το view εκτελεί τη μέθοδο \texttt{run\_utastar()}) του μοντέλου \texttt{Problem} που αντιστοιχεί στο πρόβλημα που προβάλλεται, και προβάλει τα αποτελέσματα εκτέλεσης της UTASTAR. Συγκεκριμένα εμφανίζει τον πολυκριτήριο πίνακα διατεταγμένο κατά φθίνουσα σειρά σύμφωνα με την στήλη ολικών χρησιμοτήτων (που επιστρέφει η \hyperref[sssec:utastar()]{\texttt{utastar()}}), το $ \tau $ του Kendall, τον πίνακα με τα βάρη των κριτηρίων (βάρη στην συνάρτηση αξίας του μοντέλου) και τα γραφήματα κανονικοποιημένων μερικών χρησιμοτήτων των κριτηρίων.
\item[\hypertarget{evaluatealternative}{\texttt{evaluate\_alternative}}] View υπεύθυνο για την αξιολόγηση νέας εναλλακτικής σύμφωνα με το μοντέλο UTASTAR του προβλήματος. Με τον υπολογισμό της χρησιμότητας της νέας εναλλακτικής εμφανίζει τον ταξινομημένο κατά φθίνουσα σειρά χρησιμότητας πολυκριτήριο πίνακα με την νέα εναλλακτική. Έτσι φαίνεται η κατάταξη της νέας εναλλακτικής σε σχέση με τις προϋπάρχουσες.
\item[\hypertarget{downloadmodel}{\texttt{download\_model}}] Επιτρέπει την εξαγωγή των παραμέτρων του μοντέλου, για χρήση εκτός εφαρμογής. Τα αποτελέσματα περιέχονται σε συμπιεσμένη αρχειοθήκη zip, ως αρχεία Excel. Κάθε λύση (αρχική, ενδιάμεσες κατά την ανάλυση ευαισθησίας και τελική) εμφανίζεται σε ξεχωριστό αρχείο Excel. Κάθε αρχείο λύσης περιέχει τέσσερα φύλλα: το πρώτο έχει τον πολυκριτήριο πίνακα με τις ολικές χρησιμότητες και τα σφάλματα υποεκτίμησης ($ \sigma^{+} $) και υπερεκτίμησης ($ \sigma^{-} $), το δεύτερο έχει τα βάρη της συνάρτησης αξίας, το τρίτο τις τιμές $ w_{ij} $, το τέταρτο τις μερικές χρησιμότητες των κριτηρίων και το πέμπτο τον αριθμό $ \tau $ του Kendall.
\item[\hypertarget{deleteproblem}{\texttt{delete\_problem}}] Χρησιμοποιείται από το \hyperlink{IndexView}{\texttt{IndexView}} και από το \hyperlink{replaceproblemfile}{\texttt{replace\_problem\_file}} view για τη διαγραφή προβλημάτων που δεν χρειάζονται πια.
\end{description}

\subsection{\texttt{run.py}}
Το πρόγραμμα συνοδεύεται από το εκτελέσιμο \texttt{run.py}, το οποίο είναι υπεύθυνο για την αρχικοποίηση της βάσης δεδομένων της εφαρμογής και την εκκίνηση του web server που επιτρέπει την πρόσβαση στην εφαρμογή μέσω του περιηγητή ιστού. Το αρχείο μπορεί να εκτελεστεί και εκτός του εικονικού περιβάλλοντος των \hyperref[sec:instructions]{οδηγιών χρήσης}. Με το όρισμα \texttt{--reset} το \texttt{run.py} επαναφέρει το πρόγραμμα στην αρχική του κατάσταση και διαγράφει όλα τα εισαγμένα στην εφαρμογή δεδομένα.

\subsection{Υλοποίηση Αναδράσεων}
Παρακάτω παρουσιάζεται ο τρόπος με τον οποίο το πρόγραμμα διαχειρίζεται την κάθε ανάδραση που περιγράφεται στην ενότητα \ref{ssec:minora-theory}.

\begin{description}
	\item[Ανάδραση τύπου 1]
	Οι αλλαγές στη μοντελοποίηση των κριτηρίων γίνονται στο αρχείο Excel του προβλήματος. Στο πρώτο φύλλο προσθέτονται, αφαιρούνται ή μεταβάλλονται οι στήλες των κριτηρίων και στο δεύτερο φύλλο μεταβάλλεται αντίστοιχα η μονοτονία τους.
	\item[Ανάδραση τύπου 2]
	Στη συγκεκριμένη περίπτωση δεν χρειάζεται κάποια αλλαγή στα δεδομένα του προβλήματος.
	\item[Ανάδραση τύπου 3]
    Η ανάδραση μπορεί να επιτευχθεί με τους παρακάτω τρεις τρόπους: Ο χρήστης μεταβάλλει τις εκτιμήσεις των εναλλακτικών επιλογών στα κριτήρια αλλάζοντας τις τιμές τους στις στήλες των κριτηρίων του πολυκριτηρίου πίνακα στο πρώτο φύλλο του αρχείου Excel. Επίσης μπορεί να μεταβάλλει τον αριθμό υποδιαστημάτων κάθε κριτηρίου στο δεύτερο φύλλο. Τέλος, μπορεί να αλλάξει τις τιμές $ \delta $ και $ \varepsilon $ στη σελίδα \textbf{Problem Details} και να εκτελέσει ξανά την UTASTAR.
	\item[Ανάδραση τύπου 4]
    Σε αυτήν την περίπτωση ο χρήστης μπορεί είτε να αντικαταστήσει το αρχείο Excel με τα δεδομένα του προβλήματος από τη σελίδα \textbf{Replace Problem File}, η οποία είναι διαθέσιμη και από την σελίδα των αποτελεσμάτων μέσω του κουμπιού \textit{Retry}, είτε να διαγράψει το πρόβλημα από την εφαρμογή, πατώντας το εικονίδιο του κάδου απορριμμάτων στην αρχική οθόνη \textbf{Home}, και να δημιουργήσει καινούριο μέσω της σελίδας \textbf{Upload Problem}.
	\item[Ανάδραση τύπου 5]
	Νέες εναλλακτικές επιλογές μπορούν να αξιολογηθούν βάσει του προκύπτοντος μοντέλου UTASTAR μέσω της σελίδας \textbf{Evaluate New Alternative}. Ακολουθώντας τις οδηγίες στην σελίδα και συμπληρώνοντας τα απαραίτητα πεδία υπολογίζεται και εμφανίζεται η χρησιμότητα και η κατάταξη της νέας εναλλακτικής στον πολυκριτήριο πίνακα. Σε περίπτωση ασυμφωνίας τροποποιείται το πρόβλημα σύμφωνα με τις παραπάνω αναδράσεις και εφαρμόζεται πάλι η UTASTAR.
\end{description}

\section{Οδηγίες Χρήσης}
\label{sec:instructions}

Για την εκτέλεση του προγράμματος πρέπει να υπάρχει εγκατεστημένη η γλώσσα προγραμματισμού \href{https://www.python.org/downloads/}{Python}
\footnote{\href {https://www.python.org/downloads/}{python.org/downloads/}} και το εργαλείο \href{https://pipenv.pypa.io/en/latest/install/}{Pipenv}
\footnote{\href {https://pipenv.pypa.io/en/latest/install/}{pipenv.pypa.io/en/latest/install/}}.

\begin{enumerate}
	\item Αποσυμπιέζετε το συμπιεσμένο αρχείο και ανάλογα με το λειτουργικό σύστημα που χρησιμοποιείτε ανοίγετε τη γραμμή εντολών ή το τερματικό στον φάκελο \texttt{bluehunt} και εκτελείτε τις παρακάτω εντολές για να δημιουργήσετε ένα εικονικό περιβάλλον (virtual environment) με όλα τα πακέτα (packages) που χρειάζονται για την εκτέλεση του προγράμματος.
	
	\begin{verbatim}
			pipenv install
			
			pipenv shell
	\end{verbatim}
	
	\item Εκτελείτε την εντολή \texttt{python run.py} και στη συνέχεια κατευθύνεστε με τον περιηγητή στην ιστοσελίδα \url{http://127.0.0.1:8000}. Αυτή είναι η αρχική σελίδα του προγράμματος που παρουσιάζει τη λίστα με τα προβλήματα που εισάγονται στην εφαρμογή.
	
	\begin{figure}[H]
		\centering
		\includegraphics[width=0.8\linewidth]{media/index_no_prob.png}
		\caption{Αρχική σελίδα με κανένα ανεβασμένο πρόβλημα}
		\label{fig:index_no_prob}
	\end{figure}
	
	\begin{figure}[H]
		\centering
		\includegraphics[width=0.8\linewidth]{media/index_with_prob.png}
		\caption{Αρχική σελίδα με ανεβασμένο πρόβλημα}
		\label{fig:index_with_prob}
	\end{figure}
	
	Εάν υπάρχει εισηγμένο πρόβλημα τότε πατώντας πάνω στο όνομα του, θα μεταφερθείτε στην σελίδα όπου προβάλλονται τα δεδομένα και οι παράμετροι του προβλήματος. Με το κουμπί \textbf{Results} παρουσιάζεται αμέσως η λύση του συγκεκριμένου προβλήματος ενώ με το κόκκινο κουμπί διαγράφεται το πρόβλημα από την βάση δεδομένων.
	
	Πατώντας το κουμπί \textbf{Upload new problem!} δημιουργείτε ένα καινούργιο πρόβλημα.
	
	\item Για την δημιουργία του καινούργιου προβλήματος είναι απαραίτητο να ορίσετε ένα όνομα στο πεδίο \textbf{Problem name:} και να επιλέξετε πατώντας το κουμπί \textbf{Choose File} το αρχείο της μορφής Microsoft Excel (.xls, .xlsx) όπου περιέχει τα δεδομένα. Το αρχείο Excel αποτελείται από δύο φύλλα. Στο πρώτο εμφανίζεται ο πολυκριτήριος πίνακας, που έχει στην πρώτη στήλη τα ονόματα των εναλλακτικών, στη δεύτερη 
    την αρχική διάταξή τους και στις επόμενες στήλες (μία για κάθε κριτήριο) τις τιμές των εναλλακτικών σε κάθε κριτήριο. Στο δεύτερο φύλλο παρουσιάζονται οι παράμετροι των κριτηρίων ως εξής: στην πρώτη στήλη περιέχονται τα ονόματα των κριτηρίων, στη δεύτερη η μονοτονία τους (True για αύξον και False για φθίνον κριτήριο) και στην τρίτη ο αριθμός υποδιαστημάτων στα οποία θα χωριστεί το διάστημα τιμών τους. Παραδείγματα της μορφής των φύλλων φαίνονται στις εικόνες \ref{fig:excel_sheet_1} και \ref{fig:excel_sheet_2}.
	
	\begin{figure}[H]
		\centering
		\includegraphics[width=0.8\linewidth]{media/upload.png}
		\caption{Σελίδα για μεταφόρτωση ενός προβλήματος}
		\label{fig:uplaod}
	\end{figure}
	
	\item Πατώντας το κουμπί \textbf{Create} ανακατευθύνεστε στη επόμενη σελίδα όπου παρουσιάζονται τα δεδομένα του προβλήματος που μεταφορτώθηκε ενώ μπορείτε να αλλάξετε τις παραμέτρους $δ$ και $ε$, εάν δεν συμφωνείτε με τις προεπιλεγμένες τιμές που εμφανίζονται.
	
	\begin{figure}[H]
		\centering
		\includegraphics[width=0.8\linewidth]{media/details.png}
		\caption{Σελίδα για προβολή περιεχομένων προβλήματος (συγκεκριμένα τα δεδομένα του προβλήματος των μεταφορικών μέσων)}
		\label{fig:details}
	\end{figure}

	Με το κουμπί \textbf{Submit} γίνεται επίλυση του προβλήματος με τη μέθοδο UTASTAR.
	
	\item Παρουσιάζονται τα αποτελέσματα δηλαδή ο πολυκριτήριος πίνακας με τις τελικές ολικές χρησιμότητες, το $τ$ του Kendall, οι μεταβλητές $w$ του κάθε κριτηρίου και τα διαγράμματα των κανονικοποιημένων μερικών χρησιμοτήτων κάθε κριτηρίου. 
	
	\begin{figure}[H]
		\centering
		\includegraphics[width=0.65\linewidth]{media/results.png}
		\caption{Τελικός πολυκριτήριος πίνακας, τ του Kendall και διαγράμματα}
		\label{fig:results_1}
	\end{figure}
	

	Στο τέλος υπάρχουν 3 κουμπιά με από τα οποία μπορείτε να κάνετε κάποιες ενέργειες.
	
	\begin{itemize}
		\item Με το κουμπί \textbf{Retry} αντικαθιστάτε το τρέχων αρχείο Excel με κάποιο άλλο.
		
		\begin{figure}[H]
			\centering
			\includegraphics[width=0.8\linewidth]{media/retry.png}
			\caption{Αντικατάσταση αρχείου Excel με άλλο}
			\label{fig:retry}
		\end{figure}
	
		Συμπληρώνετε τα στοιχεία σύμφωνα με το βήμα 3.
				
		\item Με το κουμπί \textbf{Download Generated Model} λαμβάνετε τα αποτελέσματα σε μορφή αρχείου Microsoft Excel. 
		\item Με το κουμπί \textbf{Evaluate New Alternative} προσθέτετε επιπλέον εναλλακτική και την αξιολογείτε (υπολογίζοντας την χρησιμότητά της) με το μοντέλο που προέκυψε από την εκτέλεση της UTASTAR στα αρχικά δεδομένα του προβλήματος.
		
		\begin{figure}[H]
			\centering
			\includegraphics[width=0.8\linewidth]{media/evaluate.png}
			\caption{Προσθήκη επιπλέον εναλλακτικής}
			\label{fig:evaluate}
		\end{figure}
		
		Στο κενό \textbf{New alternative's name:} δίνετε το όνομα της εναλλακτικής και στα κενά \textbf{Value for criterion...:} συμπληρώνετε τη τιμή της συγκεκριμένης εναλλακτικής για το συγκεκριμένο κριτήριο.	
	\end{itemize}
\end{enumerate}

\section{Αναλυτικό Παράδειγμα}
\label{sec:example}

Στα πλαίσια της παρουσίασης του παραδείγματος θα χρησιμοποιηθούν τα δεδομένα του προβλήματος επιλογής του καλύτερου μέσου μαζικής μεταφοράς.

\paragraph{Δημιουργία αρχείου Excel}
Αρχικά δημιουργούμε το αρχείο της μορφής Excel με βάση τη δομή που περιγράφεται στο βήμα 3 των οδηγιών. Στο πρώτο φύλλο υπάρχει η κατάταξη του χρήστη και οι τιμές των κριτηρίων. Έχουμε 5 διαφορετικά μέσα μεταφοράς και 3 διαφορετικά κριτήρια. Στο δεύτερο φύλλο σημειώνουμε την μονοτονία των 3 διαφορετικών κριτηρίων και των αριθμό των υποδιαστημάτων. Ανάλογα με το εάν μας ενδιαφέρει η ελαχιστοποίηση ή η μεγιστοποίηση του κριτηρίου σημειώνουμε με \textbf{False} ή \textbf{TRUE}, αντίστοιχα.

Συγκεκριμένα στο κριτήριο «Τιμή» μας ενδιαφέρει η ελαχιστοποίηση του, γι'αυτό και είναι φθίνον άρα σημειώνεται με \textbf{False}. Παρομοίως και το κριτήριο «Διάρκεια». Στο κριτήριο «Άνεση» μας ενδιαφέρει η μεγιστοποίηση του, άρα το σημειώνουμε με \textbf{True}. 

Επίσης διαλέξαμε να χωρίσουμε το κριτήριο «Τιμή» σε 2 υποδιαστήματα και τα κριτήρια «Διάρκεια» και «Άνεση» σε 3.

Το τελικό αρχείο έχει την παρακάτω μορφή.

\begin{figure}[H]
	\centering
	\includegraphics[width=0.7\linewidth]{media/excel_sheet_1.png}
	\caption{Πρώτο φύλλο Excel}
	\label{fig:excel_sheet_1}
\end{figure}

\begin{figure}[H]
	\centering
	\includegraphics[width=0.7\linewidth]{media/excel_sheet_2.png}
	\caption{Δεύτερο φύλλο Excel}
	\label{fig:excel_sheet_2}
\end{figure}

\paragraph{Χειρισμός web interface}
Ενεργοποιούμε το web interface με βάση το βήμα 1 των οδηγιών. Επιλέγουμε \textbf{Upload new problem!} και στην επόμενη σελίδα δίνουμε το όνομα Public Transportation και επιλέγουμε το αρχείο Excel που δημιουργήσαμε παραπάνω.

Πατώντας \textbf{Create} βλέπουμε τα περιεχόμενα του προβλήματος και διαλέγουμε τους συντελεστές $δ$ και $ε$. Στη συγκεκριμένη περίπτωση αφήνουμε τις προεπιλεγμένες τιμές. Δηλαδή ως κατώφλι προτίμησης $δ = 0.05$ και ως $ε = 0.01$.

Πατώντας \textbf{Submit} μεταφερόμαστε στη τελική σελίδα του προγράμματος, όπου παρουσιάζονται τα αποτελέσματα.

\paragraph{Παρουσίαση αποτελέσματος}
Όπως μπορεί να παρατηρηθεί από τον πολυκριτήριο πίνακα στη τελευταία στήλη υπάρχουν οι τιμές των χρησιμοτήτων κάθε εναλλακτικής.

\begin{figure}[H]
	\centering
	\includegraphics[width=0.65\linewidth]{media/results.png}
	\caption{Τελικός πολυκριτήριος πίνακας, τ του Kendall και διαγράμματα κανονικοποιημένων μερικών χρησιμοτήτων των κριτηρίων}
	\label{fig:results_1_1}
\end{figure}

Το $τ$ του Kendall υπολογίστηκε ίσο με $1$ και οι τιμές των μεταβλητών $w$ των κριτηρίων ίσες με $0.5063$ για το κριτήριο «Τιμή», $0.3417$ για το κριτήριο «Διάρκεια» και $0.1521$ για το κριτήριο «Άνεση». Επίσης παρουσιάζονται τα γραφήματα των κανονικοποιημένων μερικών χρησιμοτήτων κάθε κριτηρίου. 

\section{Συμπεράσματα}
\label{sec:conclusion}
Καταλήγοντας, το πρόγραμμα και η εργασία που συντάχθηκε ολοκλήρωσαν με επιτυχία τους αντικειμενικούς στόχους που τέθηκαν στην ενότητα \ref{sec:goals}. Δημιουργήθηκε πρόγραμμα σε γλώσσα προγραμματισμού Python με το οποίο ο χρήστης χρησιμοποιεί τη μέθοδο UTASTAR και τη μέθοδο MINORA για να επιλύσει το πολυκριτήριο πρόβλημα που επιθυμεί. Τα αποτελέσματα και τα δεδομένα του κάθε προβλήματος παρουσιάζονται αναλυτικά μέσω της διεπαφής που δημιουργήθηκε (web interface). Επίσης παρέχεται αναλυτική περιγραφή των περιεχομένων του προγράμματος σε Python καθώς και του web interface. Τελικά δόθηκαν οι απαραίτητες οδηγίες με τις οποίες ο χρήστης θα μπορέσει να αξιοποιήσει τις παραπάνω λειτουργίες.

\section{Βιβλιογραφία}
\label{sec:bibliography}
% Non cited entries
\nocite{utastar-lecture}
\nocite{utastar-book}
\nocite{sya-book}

\printbibliography
\end{document}