\documentclass[12pt,a4paper,titlepage]{article}

% Font related packages
\usepackage{fontspec}
\setmainfont{Lato}
\setsansfont{Lato}
\setmonofont{Fira Mono}
\newfontfamily\greekfont[Script=Greek]{Lato}

% Greek monospace font
\newfontfamily{\greekfonttt}{Fira Mono}
\usepackage{polyglossia}
\setdefaultlanguage{greek}

% Math packages
\usepackage{amsmath}
\usepackage{amsfonts}
\usepackage{amssymb}
\usepackage{makeidx}
\usepackage{eurosym}

% Graphics
\usepackage{graphicx}
\usepackage{float}
\usepackage{tikz}
\usetikzlibrary{arrows,positioning,shapes}

\usepackage{appendix} % Add appendix to document
\usepackage[margin=3cm]{geometry} % Change page margins

% Table packages
\usepackage{multirow}
\usepackage{color, colortbl}
\usepackage{array}
\usepackage{longtable}
\usepackage{lscape}

% PDF options
\usepackage[unicode,hidelinks,pdftitle={Ανάπτυξη συστήματος με υλοποίηση μεθοδολογίας MINORA (UTA) σε περιβάλλον Python με web interface},pdfauthor={Dimitrios Maroulidis, Myron Pachakis},pdfcreator={},pdfproducer={}]{hyperref}

% Colors to use in tables
\definecolor{LightGray}{gray}{0.95}
\definecolor{Gray}{gray}{0.9}
\definecolor{LightRed}{HTML}{ff6b6b}
%\definecolor{LightBlue}{HTML}{aacfff}
%\definecolor{LightBlue}{HTML}{5eaeff}
\definecolor{LightBlue}{HTML}{a7d2ff}

% Table column types for math environments
\newcolumntype{e}{>{$}c<{$}}
\renewcommand{\arraystretch}{1.2}
\usepackage{ragged2e}

% Math font
\usepackage{unicode-math}
\setmathfont{Asana Math}

% Use section based numbering for equations.
% amsmath is needed for this.
\numberwithin{equation}{section}

% Use comma as decimal point
%\usepackage{icomma}

% Code highlighting
\usepackage{minted}

% Comments
\usepackage{comment}

\author{Δημήτριος Μαρουλίδης (\textit{2018010111}) \\
    Μύρων Παχάκης (\textit{2018010028})}
\title{Ανάπτυξη συστήματος με υλοποίηση μεθοδολογίας MINORA (UTA) σε περιβάλλον Python με web interface}
\date{\today}

\begin{document}
\maketitle	
\tableofcontents
%	\listoffigures
%	\listoftables

\clearpage
% BEGIN HERE

\begin{comment}
Πλάνο Σύνταξης Αναφοράς
Εισαγωγή
1.Αντικειμενικοί Στόχοι Εργασίας-Προγράμματος
2.Περιγραφή μεθόδου θεωρητικά (Λογική της μεθόδου, ποιός την έφτιαξε)
3.Αναλύτικη Περιγραφή Μεθόδου (Χρησιμοποιούμε αυτό, κάνουμε αυτό, παρουσίαση αποτελέσματος, δεν σταματάμε όμως εκεί έχουμε ανατροφοδότηση από τον χρήστη (web interface), κάνει αυτή τη διαδικασία ο χρήστης)
4.Περιγραφή Υλοποίησης
5.Αναλυτικό Παράδειγμα εφαρμογής της μεθόδου
6.Παράρτημα -> Οδηγίες Χρήσης του προγράμματος-μεθόδου
7.Συμπεράσματα
8.Βιβλιογραφία
\end{comment}

\begin{abstract}
Η εργασία παρουσιάζει τη δημιουργία ενός αλληλεπιδραστικού προγράμματος στη γλώσσα προγραμματισμού Python για την επίλυση πολυκριτήριων συστήματων υποστήριξης αποφάσεων χρησιμοποιώντας το σύστημα MINORA. Για την υλοποίηση του συστήματος MINORA θα γίνει, επίσης, χρήση της μεθόδου πολυκριτήριας ανάλυσης UTASTAR. 

Η εργασία χωρίζεται σε δύο μέρη, το θεωρητικό αποτελείται από τα Κεφάλαια (1-3) και το προγραμματιστικό από τα Κεφάλαια (4-7). Στο Κεφάλαιο 1 παρουσιάζονται οι αντικειμενικοί στόχοι που αναμένονται από το πρόγραμμα. Στο Κεφάλαιο 2 περιγράφεται αναλυτικά η θεωρία πίσω από την μέθοδο UTASTAR και το σύστημα MINORA όπως η λογική λειτουργίας και ο τρόπος χρήσης τους, καθώς και οι πρωτοπόροι υλοποίησης και παρουσίασης των μεθόδων. Στο Κεφάλαιο 3 γίνεται αναλυτική παρουσίαση βήμα-βήμα στη εφαρμογή των μεθόδων UTA-UTASTAR-MINORA.

Στο Κεφάλαιο 4 αναλύεται διεξοδικά το πρόγραμμα σε γλώσσα Python καθώς και η αλληλεπίδραση μέσω του web interface, ενώ στο Κεφάλαιο 5 δίνεται ένα παράδειγμα πολυκριτήριων αποφάσεων και εφαρμόζεται η μέθοδος MINORA. Στο Κεφάλαιο 6 εντοπίζονται εκτενείς οδηγίες χρήσης του προγράμματος και στο Κεφάλαιο 7 συνοψίζονται τα συμπεράσματα της εργασίας και της χρήσης των μεθόδων-προγράμματος. Στο Κεφάλαιο 8 δίνεται η σχετική βιβλιογραφία και οι πηγές που χρησιμοποιήθηκαν.
\end{abstract}


%Μία πρώτη προσπάθεια είναι, δες ότι δεν σου αρέσει ή οτι θες να αλλάξουμε (πιθανών και να γραψουμε λιγα παραπανω?) Επίσης όπου υπάρχουν οι αριθμοί των κεφαλαίων θα αλλάξουν με τα tags των sectors για να μην γινεται μπερδεμα.  

\section{Στόχοι Προγράμματος}    

\section{Θεωρητική Περιγραφή Μεθόδου}
\label{sec:2}
Η μέθοδος πολυκριτήριας ανάλυσης UTASTAR είναι υπεύθυνη για την εκτίμηση μίας ή παραπάνω συναρτήσεων αξίας χρησιμοποιώντας διαδικασίες γραμμικού προγραμματισμού. Στόχος των συναρτήσεων αξίας που θα εξαχθούν είναι η όσο το δυνατόν καλύτερη κατάταξη κάποιων επιλογών $a$ σε σύγκριση με την αρχική κατάταξη που έχει δωθεί από τον χρήστη. Ο χρήστης είναι υπεύθυνος για την εισαγωγή ενός συνόλου επιλογών $A_{R}$ κατανεμημένου με σειρά προτίμησης καθώς επίσης και ενός συνόλου κριτηρίων επιλογής $g_{n}$.





\end{document}