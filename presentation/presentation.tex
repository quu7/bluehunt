\documentclass{beamer}
\usetheme{Berlin}

% Font related packages
\usepackage{fontspec}
\setmainfont{Lato}
\setsansfont{Lato}
\setmonofont{Fira Mono}
\newfontfamily\greekfont[Script=Greek]{Lato}

% Greek monospace font
\newfontfamily{\greekfonttt}{Fira Mono}
\usepackage{polyglossia}
\setdefaultlanguage{greek}

% Math packages
\usepackage{amsmath}
\usepackage{amsfonts}
\usepackage{amssymb}
\usepackage{makeidx}
\usepackage{eurosym}

% Graphics
\usepackage{graphicx}
\usepackage{float}
\usepackage{tikz}
\usetikzlibrary{arrows,positioning,shapes}

% PDF options
%\usepackage[unicode,hidelinks,pdftitle={Ανάπτυξη συστήματος με υλοποίηση μεθοδολογίας MINORA (UTA) σε περιβάλλον Python με web interface},pdfauthor={Dimitrios Maroulidis, Myron Pachakis},pdfcreator={},pdfproducer={}]{hyperref}

% Math font
\usepackage{unicode-math}
\setmathfont{Asana Math}

% Code highlighting
\usepackage{minted}

\title{Εφαρμογή Bluehunt}
\subtitle{Ανάπτυξη συστήματος με υλοποίηση μεθοδολογίας MINORA (UTA) σε περιβάλλον Python με web interface}
\author{Δημήτριος Μαρουλίδης \and Μύρων Παχάκης}
\institute{Πολυτεχνείο Κρήτης}
\date{\today}
\begin{document}
\begin{frame}[plain]
    \maketitle
\end{frame}

\section{Εισαγωγή}
\begin{frame}{Αντικείμενο της εργασίας}
    Δημιουργία συστήματος που υλοποιεί τη μεθοδολογία MINORA, γραμμένο σε γλώσσα Python με διεπαφή ιστού 
    (web interface) για αλληλεπίδραση με χρήστες.
\end{frame}
\begin{frame}{Εφαρμογή Bluehunt}
    Αποτέλεσμα της εργασίας είναι η εφαρμογή Bluehunt.
    
    \begin{itemize}
        \item Υλοποιεί τον αλγόριθμο UTASTAR.
        \item Έχει web interface βασισμένο στο πρότζεκτ Django.
        \item A
    \end{itemize}
\end{frame}
\end{document}
